\documentclass[11pt]{article}
\usepackage[margin=1in]{geometry}
\usepackage{hyperref}
\usepackage{amsmath}
\usepackage{listings}
\usepackage{enumitem}

\title{AIF Cheatsheet for clj-ants-aif}
\author{}
\date{}

\begin{document}
\maketitle

\section*{Purpose}
This is a short, technical map of the Active Inference Framework (AIF) in this repo.
It is intended for a collaborator who knows modeling and inference, but is new to this codebase.

\section*{Definitions and Core Equation}
\textbf{Notation used in this repo} (matches field names in the AIF result maps):
\begin{itemize}[leftmargin=*]
  \item $o$ --- observation (normalized sensory evidence map from \texttt{observe/g-observe}).
  \item $a$ --- action (one of \texttt{:forage}, \texttt{:return}, \texttt{:pheromone}, \texttt{:hold}).
  \item $\mu$ --- latent beliefs (map under \texttt{:mu}, including position, goal, hunger, and predicted senses).
  \item $\Pi_o$ --- observation precision (map under \texttt{:prec :Pi-o}, per-sensory channel weights).
  \item $\tau$ --- action temperature (policy precision) used in softmax over $-G/\tau$.
  \item $G$ --- expected free energy for an action (scalar per action, lower is better).
\end{itemize}

\textbf{Typical AIF policy equation} (as implemented here, in words and weights):
\begin{equation*}
G(a) = \lambda_p\,\text{risk}(a) + \lambda_a\,\text{ambiguity}(a)
      + \lambda_c\,\text{colony}(a) + \lambda_s\,\text{survival}(a)
      - \lambda_i\,\text{info}(a) + \text{action-cost}(a)
\end{equation*}
Actions are sampled via a softmax over $-G/\tau$; $\tau$ is coupled to hunger,
survival pressure, and colony reserve levels (see \texttt{ants.aif.policy}).

\section*{Where AIF Lives}
Core entry points (all pure, no world mutation):
\begin{itemize}[leftmargin=*]
  \item \texttt{src/ants/aif/core.clj} -- orchestration of observe $\to$ perceive $\to$ policy
  \item \texttt{src/ants/aif/observe.clj} -- sensory normalization into observation vectors
  \item \texttt{src/ants/aif/perceive.clj} -- predictive-coding micro-steps (belief updates)
  \item \texttt{src/ants/aif/policy.clj} -- expected free energy (EFE) evaluation and softmax action choice
  \item \texttt{src/ants/war.clj} -- scenario wiring and call site for AIF vs classic ants
\end{itemize}

\section*{Agent and World Model (Contract)}
The core contract (\texttt{AGENTS.md}):
\begin{itemize}[leftmargin=*]
  \item An ant's \emph{state} is its grid location stored in a Clojure \texttt{agent}.
  \item Mutable attributes live in the \emph{cell map} at that location:
  \texttt{@(place loc) => \{:ant \{...\} :food ... :pher ... :home ...\}}.
  \item A brain is a function \texttt{(loc) -> next-loc} that mutates the world inside a \texttt{dosync}.
\end{itemize}

AIF ant attributes (in the cell map) include:
\begin{itemize}[leftmargin=*]
  \item \texttt{:dir}, \texttt{:species :aif}, \texttt{:brain :aif}
  \item Latent state \texttt{:mu}: \texttt{\{:pos [x y] :goal [gx gy] :h hunger\}}
  \item Precision \texttt{:prec}: \texttt{\{:Pi-o \{sensory precisions\} :tau action temperature\}}
\end{itemize}

\section*{AIF Loop (per tick)}
The function \texttt{ants.aif.core/aif-step} is pure and returns a rich result map.
Pipeline:
\begin{enumerate}[leftmargin=*]
  \item \textbf{Observe} \newline
  \texttt{observe/g-observe} normalizes local evidence (food, pheromone, proximity to home, etc.)
  into a map with values in $[0,1]$.

  \item \textbf{Perceive} \newline
  \texttt{perceive/perceive} runs capped micro-steps (predictive coding). It updates
  $\mu$ (beliefs) and precision, tracks free-energy proxy, and returns a trace.

  \item \textbf{Policy} \newline
  \texttt{policy/choose-action} computes one-step expected free energy $G$ for each action,
  applies biases and constraints, and samples via softmax using temperature $\tau$.
\end{enumerate}

The output includes the updated ant and diagnostics:
\begin{lstlisting}[basicstyle=\ttfamily\small]
{:ant updated-ant
 :action chosen
 :observation observation
 :perception perception
 :policy policy
 :G ... :P ...
 :diagnostics {:need ... :dhdt ... :tau ...}}
\end{lstlisting}

\section*{Observation Model (evidence)}
\texttt{observe/g-observe} computes normalized features, including:
\begin{itemize}[leftmargin=*]
  \item \texttt{:food}, \texttt{:pher}, \texttt{:food-trace}, \texttt{:pher-trace}
  \item \texttt{:home-prox}, \texttt{:enemy-prox}, \texttt{:dist-home}, \texttt{:reserve-home}
  \item \texttt{:h} (hunger), \texttt{:ingest}, \texttt{:cargo}, \texttt{:trail-grad}, \texttt{:novelty}
  \item \texttt{:friendly-home} (1 on own nest cell), \texttt{:white?} low-signal flag
\end{itemize}
A consistent ordering is defined by \texttt{observe/sense->vector}.

\section*{Perception Model (predictive coding)}
\texttt{perceive/perceive} iterates $N$ micro-steps:
\begin{itemize}[leftmargin=*]
  \item Maintain latent state $\mu$ with \texttt{:sens} predictions for each sensory key.
  \item Compute prediction errors with precision $\Pi_o$.
  \item Update predicted sensory beliefs and hunger $h$.
  \item Track a free-energy proxy via weighted MSE of errors.
\end{itemize}
Defaults: \texttt{max-steps=5}, \texttt{alpha=0.55}, \texttt{beta=0.3} (see \texttt{perceive.clj}).

\section*{Policy and Expected Free Energy (EFE)}
Actions are abstract and local: \texttt{:forage}, \texttt{:return}, \texttt{:pheromone}, \texttt{:hold}.

For each candidate action, the policy computes:
\begin{itemize}[leftmargin=*]
  \item \textbf{Risk} from preference mismatch (hunger, ingest).
  \item \textbf{Ambiguity} via precision-weighted outcome variance.
  \item \textbf{Information gain} (novelty and trail changes).
  \item \textbf{Colony pressure} (reserves and return bias).
  \item \textbf{Survival pressure} (hunger, distance, ingest).
  \item \textbf{Action cost} (heuristic penalties).
\end{itemize}

The expected free energy is aggregated as:
\begin{equation*}
G = \lambda_p \cdot \text{risk} + \lambda_a \cdot \text{ambiguity}
    + \lambda_c \cdot \text{colony} + \lambda_s \cdot \text{survival}
    - \lambda_i \cdot \text{info} + \text{action-cost}.
\end{equation*}

Action probabilities use a softmax over $-G/\tau$, where $\tau$ is modulated by
hunger, survival pressure, and colony reserve status.

\section*{Key Configuration Knobs}
Default AIF config is in \texttt{ants.aif.core/default-aif-config}:
\begin{itemize}[leftmargin=*]
  \item Preferences: target mean/sd for hunger and ingest.
  \item Precision control: $\tau$ floor/cap, gains for reserve and survival coupling.
  \item EFE weights: \texttt{:pragmatic}, \texttt{:ambiguity}, \texttt{:info}, \texttt{:colony}, \texttt{:survival}.
  \item Mode thresholds and heuristics for action gating.
\end{itemize}

\section*{How It Connects to the Simulation}
\begin{itemize}[leftmargin=*]
  \item \texttt{ants.war/step} calls \texttt{aif-step} for AIF ants and then applies
  the chosen action to the world state (inside STM transaction).
  \item The HUD log (\texttt{ants.ui}) prints per-tick scores and a detailed AIF line
  for a designated ant (Alice, id 5).
\end{itemize}

\section*{Useful References}
\begin{itemize}[leftmargin=*]
  \item \texttt{README.md}: high-level overview and entry points.
  \item \texttt{doc/stack-baseline.md}: an annotated single-tick trace and data contracts.
  \item \texttt{tools/trace\_tick.clj}: helper to inspect a single AIF step.
\end{itemize}

\section*{Worked Example (Single Tick)}
This example mirrors the shape of a real trace (see \texttt{doc/stack-baseline.md}).
Start from a world and an ant, then run:
\begin{lstlisting}[basicstyle=\ttfamily\small]
(require '[ants.aif.core :as core])
(let [{:keys [observation perception policy action]} (core/aif-step world ant)]
  [(:food observation) (:pher observation) (:home-prox observation)
   (:h observation) (:ingest observation) action])
;; => [0.12 0.05 0.78 0.43 0.62 :return]
\end{lstlisting}

Interpretation:
\begin{itemize}[leftmargin=*]
  \item The ant is near home (\texttt{:home-prox 0.78}) with moderate hunger.
  \item Policy selects \texttt{:return}, consistent with cargo/home bias and colony pressure.
  \item The full \texttt{:policy} map includes per-action $G$ values and probabilities.
  \item \texttt{:perception} includes a micro-step trace with $\tau$ and error summaries.
\end{itemize}

\section*{Adapting AIF to Pattern-Guided Agents}
When we adapt the ant AIF loop to pattern-guided coding agents (fuclaude/fucodex),
the same variables are reused with different semantics.

\textbf{Reinterpreted variables}:
\begin{itemize}[leftmargin=*]
  \item $o$ --- observation becomes a structured summary of the session state
  (trace events, anchors, tool results, and user feedback).
  \item $a$ --- action becomes a candidate pattern choice (e.g., a pattern ID).
  \item $\mu$ --- beliefs become the agent's state about task progress, fit of
  patterns, and uncertainty about outcomes.
  \item $\Pi_o$ --- observation precision becomes trust in evidence sources
  (e.g., trace anchors vs. narrative claims).
  \item $\tau$ --- policy precision becomes confidence in pattern selection; it
  controls how strongly the agent commits to a chosen pattern.
  \item $G$ --- expected free energy becomes a score over candidate patterns,
  combining fit, risk, and information value.
\end{itemize}

\textbf{Record shapes}:
\begin{itemize}[leftmargin=*]
  \item \textbf{PSR} (Pattern Selection Record) logs the selected pattern,
  candidates considered, and forecasted outcomes.
  \item \textbf{PUR} (Pattern Use Record) logs the outcome after applying the
  pattern and updates beliefs/precision.
\end{itemize}

\textbf{Key adaptation}: the ant action loop is preserved, but the world and
action space are replaced by a pattern catalog and a session trace. This keeps
the AIF mechanics intact while grounding them in auditable evidence.

\section*{Further Reading (AIF at the Implementation Level)}

This list is curated for readers who already understand probabilistic modeling
and control, and want to understand how Active Inference is operationalised in
codebases like this one (one-step EFE, precision control, softmax policies,
predictive coding loops).

\subsection*{Foundational but Practical}

\begin{itemize}[leftmargin=*]
  \item \textbf{Karl Friston et al.} (2017).
  \emph{Active Inference: A Process Theory}.
  \newline
  This is the most implementation-oriented of Friston’s core papers.
  It introduces expected free energy, risk vs ambiguity, and the softmax
  policy update in a form close to what is implemented here.
  \newline
  \url{https://www.sciencedirect.com/science/article/pii/S0022249617300969}

  \item \textbf{Friston, Parr, de Vries} (2017).
  \emph{The Graphical Brain}.
  \newline
  Useful for understanding predictive coding as message passing with precision
  weighting, without committing to biological realism.
  \newline
  \url{https://arxiv.org/abs/1702.02297}
\end{itemize}

\subsection*{Expected Free Energy and Action Selection}

\begin{itemize}[leftmargin=*]
  \item \textbf{Parr \& Friston} (2019).
  \emph{Generalised Free Energy and Active Inference}.
  \newline
  Clarifies the decomposition of expected free energy into pragmatic
  (risk/survival) and epistemic (information gain) terms.
  This directly maps to the $\lambda$-weighted $G$ aggregation used here.
  \newline
  \url{https://arxiv.org/abs/1807.08475}

  \item \textbf{Da Costa et al.} (2020).
  \emph{Active Inference on Discrete State-Spaces}.
  \newline
  Highly recommended for readers coming from reinforcement learning.
  It shows discrete actions, categorical observations, and explicit softmax
  policies over $-G$.
  \newline
  \url{https://arxiv.org/abs/2001.07299}
\end{itemize}

\subsection*{Precision, Temperature, and Control}

\begin{itemize}[leftmargin=*]
  \item \textbf{Parr, Corcoran, Friston} (2019).
  \emph{Attention, Precision, and Active Inference}.
  \newline
  The clearest treatment of precision as a control variable rather than a
  statistical nuisance. Useful for understanding $\Pi_o$ and $\tau$ as
  dynamically modulated parameters.
  \newline
  \url{https://www.sciencedirect.com/science/article/pii/S1053811919306287}

  \item \textbf{Schwartenbeck et al.} (2015).
  \emph{Exploration, Novelty, Surprise, and Free Energy}.
  \newline
  A good reference for the epistemic (information-seeking) term in $G$,
  especially when novelty heuristics are used.
  \newline
  \url{https://www.frontiersin.org/articles/10.3389/fpsyg.2015.01209/full}
\end{itemize}

\subsection*{Code-Oriented and Comparative}

\begin{itemize}[leftmargin=*]
  \item \textbf{pymdp} (Parr et al.).
  \emph{A Python Library for Active Inference}.
  \newline
  Not because this repo follows it, but because it makes design choices explicit
  (factorised states, discrete policies, one-step vs multi-step planning).
  Useful for comparison.
  \newline
  \url{https://github.com/infer-actively/pymdp}

  \item \textbf{Buckley et al.} (2017).
  \emph{The Free Energy Principle for Action and Perception}.
  \newline
  A compact survey that situates AIF relative to RL and control theory.
  Good for collaborators who want orientation without metaphysics.
  \newline
  \url{https://arxiv.org/abs/1705.10033}
\end{itemize}

\subsection*{Bridges to Pattern-Guided or Cognitive Agents}

\begin{itemize}[leftmargin=*]
  \item \textbf{Friston et al.} (2021).
  \emph{Planning as Inference}.
  \newline
  Helpful for understanding why AIF ports cleanly from spatial agents
  (ants) to abstract action spaces (patterns, plans, code moves).
  \newline
  \url{https://www.sciencedirect.com/science/article/pii/S0893608021000670}

  \item \textbf{Tschantz et al.} (2020).
  \emph{Learning Action-Oriented Models through Active Inference}.
  \newline
  Useful when extending from fixed heuristics to learned priors or pattern
  preferences.
  \newline
  \url{https://arxiv.org/abs/2006.09003}
\end{itemize}

\paragraph{Reading strategy.}
For collaborators working on this repo, a good sequence is:
\newline
(1) Da Costa et al. (discrete AIF),
(2) Parr \& Friston (EFE decomposition),
(3) Attention/precision paper,
then skim pymdp to calibrate implementation choices.

\end{document}
